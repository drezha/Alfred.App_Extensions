\documentclass[a4paper,oneside]{book}

\usepackage{graphicx} % Allows insert of graphics
\usepackage{longtable} % Allows for multipage spanning tables
\usepackage{pdfpages} % Simplifies insertion of multipage PDF's
\usepackage{natbib} % Puts in bibliography style
\usepackage[pdfborder=0 0 0]{hyperref} % Adds PDF links
\usepackage{mathptm} % Changes font
\usepackage{fancyhdr} % Allows the use of the fancy Header package
\usepackage{url} % Makes URL's appear nicer and follow formatting rules
\usepackage[a4paper,vmargin={25.4mm,25.4mm},hmargin={25.4mm,25.4mm}]{geometry} % Allows the chaging of page borders

\setlength{\parindent}{0.0in} % Sets paragraph indentation to 0
\pagestyle{fancy}
\bibpunct{(}{)}{,}{a}{,}{,} % Defining the citation style
\newcommand{\degree}{\ensuremath{^\circ}} % Sets new command - Inserts degree symbol
\newcommand{\HRule}{\rule{\linewidth}{0.5mm}} % Set new command - Add blank line


\begin{document}
\chapter{Methodology}
\section{Literature Review}
The initial step of this research was to first identify an aspect of fire engineering that had not been investigated and identify an area that further research was required for.
\section{Questionnaire}
The initial step of the research was to conduct a questionnaire for people within fire engineering consultancy and those connected to the fire engineering industry, such as architects and building control authorities.
\\
\\
This questionnaire would highlight the issues currently within fire engineering with respect to BS 9999 and would hopefully bring to light problems that haven't yet been considered.
\\
\\
The questionnaire was initially designed to be completed by just fire engineering consultants within engineering firms. However after a brief pilot study, it was found that the number of fire engineers within the UK was small and with an average response rate, the numbers of completed questionnaires would be statistically insignificant. Therefore to get  more data and a representative sample of data across the whole construction industry, the questionnaire was changed and sent out to both architects, building control representatives and approved inspectors.
\\
\\
Questionnaires were sent to the local building authorities were it was expected that the building work present would encompass the use of BS 9999, such as control authorities in London and Birmingham. The full list of who received the questionnaire and the outcome is shown below in Table \ref{tab:Build_Auth}.
\begin{table}
\begin{tabular}{|c|c|}
\hline
\textbf{Building Control Authority} & \textbf{Outcome of Contact} \\
\hline
Leicester & Questionnaire Sent \\
\hline
\end{tabular}
\caption{Outcomes of emails to Building Control}
\label{tab:Build_Auth}
\end{table}
\\
The architects that were contacted for the questionnaire were chosen due to the large size of the firms in question. It was assumed that these larger firms would have had a greater amount of experience with larger projects where BS 9999 is more effective in it's use. It was also assumed that the response rate would be better as there are more architects in these companies and thus multiple architects within the same firm could reply.
\\
\\
With the fire engineering consultants, questionnaires were sent to various firms, however others were interviewed in the form of a structured interview, using the questionnaire as a basis for these interviews. The interviews will be covered in greater detail in section \ref{sec:Interviews}. The fire engineering consultants used in both the interview and questionnaire were chosen as they advertised the fact that they workedon fire engineering work. From the writers experience, some engineering firms do not advertise fire engineering consultancy to the public and keep the consultancy ``in house''. Therefore this made it slightly harder to find consultants to help with the research. Where possible, interviews were conducted with consultants as this would provide extra information over that in the questionnaires and it is hoped that this extra information and more in depth answer will prove to be more beneficial than a simple questionnaire.
\\
\section{Interviews}
\label{sec:Interviews}



\end{document}